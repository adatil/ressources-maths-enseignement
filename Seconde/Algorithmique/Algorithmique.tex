%\documentclass[a4paper,10pt]{report}\input{../0.Préambule/Préambule_cours_prof}\input{0_Préambule_Info}\begin{document}

\documentclass{article} %DÉSACTIVER POUR A5
%\documentclass[a5paper]{article} %ACTIVER POUR A5

%########
% Packages #
%########

\usepackage[utf8]{inputenc}
\usepackage[T1]{fontenc}
\usepackage[french]{babel}

%######Affichage des maths
\DecimalMathComma %pour ne plus avoir d'espace après la virgule dans l'écriture décimale des nombres

\usepackage{amsmath}
\usepackage{amssymb,amsthm}
\usepackage{mathrsfs}
\usepackage{amsopn}

\usepackage[np]{numprint}%écriture des nombres avec des espaces et en écriture scientifique

\usepackage{dsfont} %Pour faire le 1 double barre de la fonction caractéristque dans un enironnement maths. \mathds{1}
%\usepackage{bbold} %Double barre mais en petit pour tout les nombres dans un environnement maths.\mathbb{1}


%######Graphique
\usepackage[dvipsnames]{xcolor}
\usepackage{graphicx}
\usepackage{pgf}
\usepackage{tikz}
\usepackage{tkz-tab}
\usetikzlibrary{shapes,arrows}

\usepackage{geometry} 
\geometry{hscale=0.85,vscale=0.85,centering}

%######Tableau
\usepackage{array} %pour centrer dans un tableau
\usepackage{colortbl} %pour colorier les cellules lignes colonnes d'un tableau: \rowcolor{}, \columncolor{}, \cellcolor{purple!25}
\usepackage{tabularx} %quelques améliorations de l'environnement tabular
\usepackage{diagbox} %Pour faire une diagonale dans une case d'un tableau: \diagbox{bas gauche}{haut droit}
\usepackage{multirow} %fusionner des cellules horizontalement

%######Hyperliens dans les pdf

\usepackage[colorlinks=true,linkcolor=magenta,urlcolor=magenta]{hyperref}% Pour créer des liens à l'intérieur du pdf: \hyperlink{label}{texte du lien} permettra d'atteindre la cible identifiée par \hypertarget{label}{texte de la cible}. Les textes du lien et de la cible peuvent être vides.

%######Des symboles et images

\usepackage{marvosym} %Image de téléphone protable avec la commande \Mobilefone

\usepackage{fdsymbol} %Notamment le cœur plein: \varheartsuit

\usepackage{eurosym}%pour afficher le symbole euro

%######Vrac

\usepackage{enumerate}%énumération avec des lettres 
\usepackage{tasks}%Pour avoir une liste en ligne utiliser \begin{tasks}(2) (pour deux colonnes) et non pas enumerate puis \task et non pas \item 
	
	\usepackage{stmaryrd}%pour faire des "intervalles" d'entiers \llbracket et \rrbracket
	
	\usepackage{xlop}%poser les calculs en colonne: \opdiv[displayintermediary=nonzero,voperation=top,shiftdecimalsep=none]{27}{45}
	\opset{decimalsepsymbol={,}}
	
	\usepackage{verbatim}%pour utiliser commande \exclure et normalement pour faire l'affichage tel quel sans compiler le texte. 
	%\usepackage{alltt}%Pour utiliser une commande latex dans un environnement verbatim il faut utiliser: alltt
	%Pour écrire juste suelques mots en verbatim au milieu d'un phrase: \verb|quelques mots|
	
	\usepackage{fancyhdr}
	
	%######Algo
	
	\usepackage{listings} % \begin{lstlisting} \end{lstlisting} affiche du code comme le fait le langage choisi. \lstset{language=Pascal} \lstset{language=Python} pour choisir le langage dans le document avant chaque programme ou avant le \begin{document} pour l'appliquer à tout le document. 
		%\lstset{} permet d'indiquer toutes les options. Pas de caractère accentué (option lourdingue à rajouter) qui vont s'ppliquer pour toute la suite du document: \lstset{language=Python}
		%Il espossible d'inclure un code python d'un fichier extérieur \lstinputlisting{source_filename.py}.
		%Il est possible de définir une présentation personnalisé par un ensemble de configuration enregistré dans un fichier de style
		\lstdefinestyle{pythonstyle}{
			language=Python,
			backgroundcolor=\color{gray!30},   
			commentstyle=\color{Plum},
			keywordstyle=\color{blue},
			numberstyle=\tiny\color{black},
			stringstyle=\color{ForestGreen},
			basicstyle=\ttfamily\color{black},
			breakatwhitespace=false,         
			breaklines=true,                 
			captionpos=b,                    
			keepspaces=true,                 
			numbers=none,                   
			numbersep=5pt,                  
			showspaces=false,                
			showstringspaces=false,
			showtabs=false,                  
			tabsize=1
		}
		\lstset{style=pythonstyle}
		
		\lstdefinestyle{bashstyle}{
			language=bash,
			backgroundcolor=\color{black},   
			commentstyle=\color{white},
			keywordstyle=\color{magenta},
			numberstyle=\tiny\color{black},
			stringstyle=\color{white},
			basicstyle=\ttfamily\footnotesize\color{white},
			breakatwhitespace=false,         
			breaklines=true,                 
			captionpos=b,                    
			keepspaces=true,                
			numbers=left,                    
			numbersep=5pt,                  
			showspaces=false,                
			showstringspaces=false,
			showtabs=false,                  
			tabsize=1
		}
		%\lstset{style=bashstyle}
		
		\usepackage[french]{algorithm2e}%pseudocode
		
		\usepackage{scratch3}
		
		%############### Formule developpée molécule chimie
		
		\usepackage{chemfig}
		
		%#####################
		% Commande et environnement #
		%#####################
		
		\theoremstyle{plain}
		
		%Pour redéfinir les commande section (changer la couleur centrer):
		\usepackage{titlesec}
		\titleformat{\section}[hang]{\color{blue}\Large\bfseries}{\thesection}{1em}{}
		\titleformat{\subsection}[hang]{\color{purple}\large\bfseries}{\thesubsection}{1em}{}
		\titleformat{\subsubsection}[hang]{\color{RoyalBlue} \bfseries}{\thesubsubsection}{1em}{}
		\titleformat{\paragraph}[hang]{}{}{1em}{}
		\titleformat{\part}[hang]{\color{blue}\Large\bfseries}{\thepart}{1em}{}
		
		\renewcommand{\thesection}{\color{blue}\Alph{section}}
		\renewcommand{\thesubsection}{\color{purple}\Roman{subsection}}
		\renewcommand{\thesubsubsection}{\color{RoyalBlue}\arabic{subsubsection}}
		\renewcommand{\thepart}{\color{purple}\Alph{part}}
		\newenvironment{correction}{\color{Brown}}{\medskip}
		
		\newenvironment{sujet}{}{\medskip}
		
		%environnement bareme
		\newenvironment{bareme}{\color{RoyalBlue}\footnotesize \hfill }{\footnotesize \emph{~points}}
		
		%environnement détais du barème
		\newenvironment{details}{\color{RoyalBlue}\noindent ~\\}{~\\}
		
		%environnement notabene
		\newenvironment{notabene}{\color{PineGreen}\noindent ~\\}{~\\}
		
		%environnement exemples
		\newenvironment{exemples}{\color{blue} \medskip \noindent \underline{Exemples.}}{}
		
		%environnement remarques
		\newenvironment{remarques}{\medskip \noindent {\color{BlueViolet}\underline{Remarques.}}\color{BlueViolet}}{}
		
		\newenvironment{lecon}{\color{black}}{}
		
		\newenvironment{culturegenerale}{\color{Violet}}{}
		
		
		%Pour redéfinir les environnements exercices et autres avec de la couleur
		\newsavebox{\selvestebox}
		\newenvironment{colbox}[1]
		{\newcommand\colboxcolor{#1}%
			\begin{lrbox}{\selvestebox}%
				\begin{minipage}{\dimexpr\columnwidth-2\fboxsep\relax}}
				{\end{minipage}\end{lrbox}%
			\begin{center}
				\colorbox{\colboxcolor}{\usebox{\selvestebox}}
		\end{center}}
		
		%environnement exercice
		\newcounter{Exercice}
		\setcounter{Exercice}{1}
		\newcounter{Exercicecorrection}
		\newenvironment{exercice}[1]{ \setcounter{Exercicecorrection}{\theExercice} \color{black} \begin{colbox}{LimeGreen!30} \hfill {\color{OliveGreen}Exercice \theExercice. {\color{black}#1}} \hfill \addtocounter{Exercice}{1}}{ \end{colbox} }
		
		%environnement exercicecorrection
		\newenvironment{exercicecorrection}{\medskip \small \color{Brown} \noindent \underline{Correction de l'exercice \theExercicecorrection}
			
		}{}
		
		%environnement definition
		\newcounter{Definition}
		\setcounter{Definition}{1}
		\newenvironment{definition}{\medskip \noindent {\color{orange}definition \theDefinition} \addtocounter{Definition}{1} \newline \noindent \begin{tabular}{|m{12cm}|}\hline \\ }{\\ \hline \end{tabular}}
		
		%environnement théorème il est possible d'ajouter un titre de théorème en mettant entre accolade le titre après le begin{theoreme}
		\newcounter{Theoreme}
		\setcounter{Theoreme}{1}
		\newenvironment{theoreme}[1]{\medskip \noindent {\color{purple}Théorème \theTheoreme #1} \addtocounter{Theoreme}{1} 
			
			\noindent \begin{tabular}{|m{12cm}|}\hline \\ }{\\ \hline \end{tabular}}
		
		%environnement proposition
		\newcounter{Proposition}
		\setcounter{Proposition}{1}
		\newenvironment{proposition}[1]{\medskip \noindent {\color{PineGreen}Proposition \theProposition #1} \addtocounter{Proposition}{1}
			
			\noindent \begin{tabular}{|m{12cm}|}\hline \\ }{\\ \hline \end{tabular}}
		
		%environnement propriété
		\newcounter{Propriete}
		\setcounter{Propriete}{1}
		\newenvironment{propriete}[1]{\medskip \noindent {\color{PineGreen}Propriété \thePropriete #1} \addtocounter{Propriete}{1}
			
			\noindent \begin{tabular}{|m{12cm}|}\hline \\ }{\\ \hline \end{tabular}}
		
		%environnement méthode
		\newcounter{Methode}
		\setcounter{Methode}{1}
		\newenvironment{methode}[1]{\medskip \noindent {\color{PineGreen}Méthode \theMethode #1} \addtocounter{Methode}{1}
			
			\noindent \begin{tabular}{|m{12cm}|}\hline \\ }{\\ \hline \end{tabular}}
		
		%environnement lemme
		\newcounter{Lemme}
		\setcounter{Lemme}{1}
		\newenvironment{lemme}[1]{\medskip \noindent {\color{PineGreen}Lemme \theLemme #1} \addtocounter{Lemme}{1}
			
			\noindent \begin{tabular}{|m{12cm}|}\hline \\ }{\\ \hline \end{tabular}}
		
		%environnement corollaire
		\newcounter{Corollaire}
		\setcounter{Corollaire}{1}
		\newenvironment{corollaire}[1]{\medskip \noindent {\color{PineGreen}Corollaire \theCorollaire #1} \addtocounter{Corollaire}{1} 
			
			\noindent \begin{tabular}{|m{12cm}|}\hline \\ }{\\ \hline \end{tabular}}
		
		%environnement démonstration
		\newcounter{Demonstration}
		\setcounter{Demonstration}{1}
		\newenvironment{preuve}[1]{\medskip \noindent {\color{PineGreen} Démonstration} \hfill #1 \addtocounter{Demonstration}{1} \color{violet} 
			
		}{\hfill $\blacksquare$}
		
		%environnement conclusion encadré et coloré
		\newenvironment{conclusion}
		{\color{PineGreen}\begin{tabular}{|c|}\hline \\ \begin{minipage}{0.85\linewidth} \begin{center} }
					{\end{center} \end{minipage} \\ \\ \hline \end{tabular} }
		
		%Commande pour l'objectif et l'écrire en vert
		\newcommand{\objectif}[1]{{\color{PineGreen}#1}
			
			\medskip}
		
		%########################
		%Test conditionnel pour l'affichage    #
		%########################
		\newif\ifs
		%\strue%affiche la boite à trous
		\sfalse%affiche la réponse
		
		%Pour faire une case à trou complétable sur le pdf
		\newcounter{Trous}
		\setcounter{Trous}{1}
		\newcommand{\trous}[2][3cm]{
			\ifs
			\begin{Form}
				\TextField[name=\theTrous ,bordercolor=,borderwidth=0, backgroundcolor=gray!20, align=1,  width=#1 ,height=0.2cm, bordersep=0,color=black] {}
			\end{Form}
			\xspace
			\else
			#2
			\fi
			\addtocounter{Trous}{1}
		}
		
		%#########################
		%en tête puis pied de page
		%#########################
		
		\pagestyle{empty}
		\pagestyle{fancy} 
		\renewcommand{\headrulewidth}{0pt}%Pas de ligne horizontale en haut
		\lhead[]{}%entre crochets pages paires entre accolades pages impaires
		%\chead[\small ]{\footnotesize \href{http://unemainlavelautre.net/2ieme.html}{Techniques 01. Calcul numérique, évaluer une expression littérale.} }% l left, c center, r right
		\rhead[]{}
		\lfoot[]{}
		\cfoot[\small -\thepage -]{\small -\thepage -}
		\rfoot[]{}
		
		%############################
		%les environnements qu'on affiche ou pas  #
		%############################
		
		\newcommand{\exclure}[1]{\renewenvironment{#1}{\begingroup\comment}{\endcomment\endgroup\ignorespaces}}
		
		%Pour cours corrections
		%\exclure{preuve} \exclure{exemples} \exclure{remarques} \exclure{corollaire} \exclure{lemme} \exclure{proposition} \exclure{theoreme} \exclure{culturegenerale} \exclure{definition} \exclure{lecon} \exclure{sujet} \exclure{notabene} \exclure{details} \exclure{bareme} \exclure{sujet} \exclure{correction}  \exclure{methode}
		
		%Pour cours exercices
		%\exclure{preuve} \exclure{exemples} \exclure{remarques} \exclure{corollaire} \exclure{lemme} \exclure{proposition} \exclure{theoreme} \exclure{culturegenerale} \exclure{definition} \exclure{lecon} \exclure{exercicecorrection} \exclure{notabene} \exclure{details} \exclure{bareme} \exclure{sujet} \exclure{correction}  \exclure{methode}
		
		%Pour cours lecon
		%\exclure{preuve} \exclure{exemples} \exclure{remarques} \exclure{lecon} \exclure{exercice} \exclure{exercicecorrection} \exclure{notabene} \exclure{details} \exclure{bareme} \exclure{sujet} \exclure{correction}
		
		%Pour cours intégrale
		\exclure{details} \exclure{bareme} \exclure{sujet} \exclure{notabene} \exclure{exercicecorrection}
		
		%Pour devoir surveillé sujet
		%\exclure{preuve} \exclure{exemples} \exclure{remarques} \exclure{corollaire} \exclure{lemme} \exclure{proposition} \exclure{theoreme} \exclure{culturegenerale} \exclure{definition} \exclure{lecon} \exclure{exercicecorrection} \exclure{notabene} \exclure{details} \exclure{correction}
		
		%Pour devoir surveillé correction
		%\exclure{preuve} \exclure{exemples} \exclure{remarques} \exclure{corollaire} \exclure{lemme} \exclure{proposition} \exclure{theoreme} \exclure{culturegenerale} \exclure{definition} \exclure{lecon} \exclure{exercicecorrection} \exclure{notabene} \exclure{sujet}
		
		%Pour devoir surveillé intégrale
		%\exclure{preuve} \exclure{exemples} \exclure{remarques} \exclure{corollaire} \exclure{lemme} \exclure{proposition} \exclure{theoreme} \exclure{culturegenerale} \exclure{definition} \exclure{lecon} \exclure{exercicecorrection} \exclure{notabene}
		
		%###############################
		%#Double numérotation des pages#
		%###############################
		%\pagenumbering{roman} %À mettre juste avant \begin{document}. DOnc simplement décommenter.
			%\pagenumbering{arabic} %À copier décommenté 
			\title{Algorithmique : Débuter en Python}
			\date{}
			\begin{document}
				\maketitle
\section{Quelques rappels}

\subsection{Qu'est-ce qu'un algorithme ?}

\begin{definition}
Un algorithme est un enchaînement d'étapes ou d'instructions à effectuer dans un certain  ordre et dont la réalisation va permettre la résolution d'un problème donné.
\end{definition}
\medskip

\noindent Le mot algorithme vient du nom du mathématicien persan al-Khuw\={a}rizm\={\i} (début du \MakeUppercase{\romannumeral9}\ieme{} siècle). Dans un livre, il décrivait des procédés de calcul à suivre étape par étape pour résoudre des problèmes ramenés à des équations.
\medskip

\noindent Un algorithme doit être lisible par tous. On l'écrit en langage courant ou langage naturel.
\medskip

\noindent Un des intérêts est de pouvoir coder (traduire) un algorithme dans un langage informatique afin qu'une machine ( ordinateur, calculatrice \ldots) puisse l'exécuter rapidement et efficacement.

\subsection{Qu'est-ce qui constitue un algorithme ?}

\begin{tabular}{p{3.7cm}p{0.2cm}p{12cm}}
\textit{Un début et une fin} &  & Les algorithmes sont constitués d'un nombre fini d'étapes à exécuter dans un ordre bien défini, on peut donc considérer qu'il y a un début et une fin. \\
\textit{Des instructions} &  & Durant l'enchaînement des étapes, ces étapes vont agir. \par On va les appeler des instructions.\\
\textit{Des variables} &  & Durant l'exécution d'un algorithme, on va avoir besoin de stocker des données, voire des résultats. Pour cela on utilise ce qu'on appelle des \og variables \fg{}. \par Le contenu d'une variable peut être modifié au cours du déroulement de l'algorithme.\\
\end{tabular}
 
\subsection{Comment analyser un algorithme ?}

Pour analyser ce que fait un algorithme on peut construire un \og tableau d'étapes \fg{}.

\begin{definition}
Un tableau d'étapes contient en première ligne (ou colonne) toutes les variables et dans  les lignes (ou colonnes) suivantes toutes les étapes d'exécution (dans l'ordre) des instructions de l'algorithme.
\end{definition}
\medskip

\begin{exemples}\\
	
On donne l'algorithme suivant où $ A $ et $ B $ désignent deux nombres réels :
\medskip

\begin{itemize}
\item[•] Calculer $A + 2B$ et remplacer $B$ par le résultat obtenu (on notera $B \gets A +  2B$)\medskip
\item[•] Calculer $B-A$ et remplacer $A$ par le résultat obtenu (on notera $A \gets B -A$)
\end{itemize}
\medskip

\begin{minipage}{8cm}
\begin{enumerate}
\item Quelles sont les variables ?\medskip
\item Faire fonctionner cet algorithme avec $A = 8$ et $B = 5$, en remplissant le tableau suivant :\medskip

{\renewcommand{\arraystretch}{1.3}
\begin{tabular}{|c|c|c|}
\cline{2-3} 
\multicolumn{1}{c|}{} & Contenu de $A$ & Contenu de $B$ \\ 
\hline 
Entrée & 8 & 5\\\hline 
Étape 1 & &\\\hline 
Étape 2 & &\\\hline 
\end{tabular}}

\end{enumerate}
\end{minipage}
\hspace{1mm}
\vline
\begin{minipage}{8cm}
\begin{enumerate}
\setcounter{enumi}{2}
\item Essayer avec d'autres valeurs du couple $(A, B)$.\medskip 

{\renewcommand{\arraystretch}{1.3}
\begin{tabular}{|c|c|c|}
\cline{2-3} 
\multicolumn{1}{c|}{} & Contenu de $A$ & Contenu de $B$ \\ 
\hline 
Entrée &&\\ \hline 
Étape 1 &&\\ \hline 
Étape 2 &&\\ \hline 
\end{tabular}} 
\medskip
          
\item Quelle valeur est contenue dans $ A $ à la fin de l'exécution de l'algorithme ? Justifier.      
\end{enumerate}
\end{minipage}
\end{exemples}



\section{Découvrir l'affectation}

\noindent L'affectation consiste à attribuer une valeur à une variable.

\noindent En langage courant, on écrit : Affecter à \textit{variable} la valeur \textit{valeur} ou \textit{calcul}.

\noindent Par exemple : Affecter à $ A $ la valeur $ 5 $.

\noindent On utilisera la notation suivante par la suite : $ A \gets 5 $.

\noindent \textbf{Exercice 1 :}

On considère l'algorithme suivant où $ A $ désigne un nombre entier relatif :

\medskip

\begin{center}
\begin{tabularx}{0.2\linewidth}{|c|X|}
\hline
ligne 1 & $ B \gets 5 $ \\
ligne 2 & $ C \gets A \times B $ \\
ligne 3 & $ A \gets C+4 $ \\
\hline
\end{tabularx}
\end{center}

\begin{enumerate}
\item \begin{enumerate}
      \item Quelle est la valeur de $ C $ à la fin de l'exécution de l'algorithme lorsque $ A = 3 $ ?
      \item Même question lorsque $ A = 10 $.
      \end{enumerate}
\item \begin{enumerate}
      \item Quelle est la valeur de $ A $ à la fin de l'exécution de l'algorithme lorsque $ A = 8 $ ?
      \item Même question lorsque $ A = -7 $.
      \end{enumerate}
\item Quelle valeur faut-il saisir au départ pour obtenir $ A = 59 $ en sortie à la fin de l'exécution de l'algorithme ?
\end{enumerate}

\medskip

\noindent \textbf{Exercice 2 :}

Un commerçant accorde une remise sur des articles. On souhaite connaître le montant de la remise en euros.

Voici un algorithme donnant la solution au problème :

$ A $ désigne le prix de départ et $ P $ la pourcentage de remise.

\medskip

\begin{center}
\renewcommand{\arraystretch}{2.1}
\begin{tabularx}{0.3\linewidth}{|c|X|}
\hline
ligne 1 & $ R \gets A \times \dfrac{P}{100} $  \\
\hline
\end{tabularx}
\end{center}

\begin{enumerate}
\item \begin{enumerate}
      \item Quelle est la valeur de $ R $ à la fin de l'exécution de l'algorithme lorsque $ A = 56 $ et $ P = 30 $ ?
      \item Donner une interprétation concrète du résultat précédent.
      \end{enumerate}
\item Même question avec $ A = 13 $ et $ P = 45 $.
\item Compléter l'algorithme pour déterminer également le prix à payer $ B $.
\item \begin{enumerate}
      \item Quelles sont les valeurs de $ R $ et $ B $ à la fin de l'exécution de l'algorithme lorsque $ A = 159 $ et $ P = 24 $ ?
      \item Donner une interprétation concrète des résultats précédents.
      \end{enumerate}
\end{enumerate}

\medskip

\noindent \textbf{Exercice 3 :}

On considère l'algorithme suivant où $ x $ désigne un nombre réel :

\medskip

\begin{center}
\begin{tabularx}{0.2\linewidth}{|c|X|}
\hline
ligne 1 & $ a \gets x^2+1 $ \\
ligne 2 & $ b \gets 2a-3 $ \\
\hline
\end{tabularx}
\end{center}

\medskip

Faire fonctionner l'algorithme en complétant le tableau qui suit.

\medskip

{\renewcommand{\tabularxcolumn}[1]{%
>{\centering\arraybackslash}m{#1}}
\begin{tabularx}{0.95\linewidth}{|c|X|X|X|X|X|}
\hline
Valeur de $ x $ avant l'exécution de l'algorithme & 3 & 4 & 7 & 10 & 20 \\
\hline
Valeur de $ a $ après l'exécution de l'algorithme & & & & & \\
\hline
Valeur de $ b $ après l'exécution de l'algorithme & & & & & \\
\hline
\end{tabularx}
}   

\medskip

\noindent \textbf{Exercice 4 :}

On considère l'algorithme suivant :

\medskip

\begin{center}
\begin{tabularx}{0.2\linewidth}{|c|X|}
\hline
ligne 1 & $ x \gets 2 $ \\
ligne 2 & $ a \gets x-1 $ \\
ligne 3 & $ b \gets 2a $ \\
ligne 4 & $ c \gets \dfrac{b}{2} $ \\
ligne 5 & $ d \gets c+2 $ \\
\hline
\end{tabularx}
\end{center}

\begin{enumerate}
\item Quelle est la valeur de $ d $ à la fin de l'exécution de l'algorithme ?
\item \begin{enumerate}
      \item Même question dans les cas suivants : on modifie la ligne 1 et on affecte à $ x $ les valeurs $ -4 $; 0; 5; 10 et 11.
      \item Que constate-t-on ? Démontrer ce résultat.
      \end{enumerate}
\end{enumerate}

\medskip

\noindent \textbf{Exercice 5 :}

Rédiger un algorithme utilisant au moins 3 variables et dont le résultat est le triple du nombre saisi au départ.

\vskip5cm

\section{Programmer l'affectation}

Syntaxe des instructions utiles dans cette fiche :

\medskip

{\renewcommand{\tabularxcolumn}[1]{%
>{\centering\arraybackslash}m{#1}}

\begin{center}
\begin{tabularx}{0.65\linewidth}{|X|X|}
\hline 
\textbf{Langage naturel} & \textbf{Python} \\ 
\hline 
Affecter à A la valeur 5 & A = 5 \\ 
\hline 
Saisir $ x $ & def nom\_{}fonction($ x $) \par \textit{Dans la console, on écrira :} \par nom\_{}fonction($ \ldots $) \\ 
\hline 
Afficher A & return A \par \textit{si une fonction a été définie comme ci-dessus} \\
\hline
Afficher A & print(A) \\
\hline
\end{tabularx}
\end{center}
}

$ ^*{}^*{} $ permet d'écrire en exposant. Par exemple, pour $ x^2 $, on écrit $ x^*{}^*{}2 $.

\textbf{Exercice 1 :}

On considère l'algorithme suivant où $ A $ désigne un nombre réel :

\medskip

\begin{tabularx}{0.2\linewidth}{|c|X|}
\hline
ligne 1 & $ A \gets 2 $ \\
ligne 2 & $ B \gets 2 \times A $ \\
ligne 3 & $ C \gets B^2 $ \\
\hline
\end{tabularx}

\medskip

Cet algorithme peut se traduire en langage de programmation Python : 

\medskip

\begin{center}
\begin{minipage}{4cm}
\begin{lstlisting}
A = 2
B = 2*A
C = B**2
print(C)
\end{lstlisting}
\end{minipage}
\end{center}

\begin{enumerate}
\item Quelle valeur de $ C $ obtient-on à la fin de l'exécution de l'algorithme ? 

      Vérifier en saisissant le programme sur Python.
\item Modifier le programme en affectant à $A$ la valeur $4$ et en affichant également la valeur de $B$ en sortie. 
      
      Tester le programme et noter les valeurs obtenues en sortie.
\item Modifier la première ligne du programme pour obtenir $C = 25$ en sortie.
\end{enumerate}

\textbf{Exercice 2 :}

\begin{enumerate}
\item Programmer avec Python chacun des algorithmes suivants. On recopiera les
programmes saisis sur la copie.
      
      \begin{minipage}{6cm}
      \begin{tabularx}{0.7\linewidth}{|c|X|}
      \hline
      ligne 1 & $ A \gets 7 $ \\
      ligne 2 & $ B \gets 6 \times A $ \\
      ligne 3 & $ C \gets A+B $ \\
      ligne 4 & $ D \gets B-C $ \\
      ligne 5 & Afficher $ D $ \\
      \hline
      \end{tabularx}
      \end{minipage}
      \begin{minipage}{6cm}
      \begin{tabularx}{0.7\linewidth}{|c|X|}
      \hline
      ligne 1 & $ M \gets 2 $ \\
      ligne 2 & $ N \gets 4 $ \\
      ligne 3 & $ A \gets M \times N $ \\
      ligne 4 & $ B \gets M+N $ \\
      ligne 5 & $ C \gets A/B $ \\
      ligne 6 & Afficher $ C $ \\
      \hline
      \end{tabularx}
      \end{minipage}
      \begin{minipage}{6cm}
      \begin{tabularx}{0.7\linewidth}{|c|X|}
      \hline
      ligne 1 & $ A \gets -1 $ \\
      ligne 2 & $ B \gets 6 $ \\
      ligne 3 & $ C \gets B^{A} $ \\
      ligne 4 & $ D \gets C^{A} $ \\
      ligne 5 & Afficher $ C $ \\
      ligne 6 & Afficher $ D $ \\
      \hline
      \end{tabularx}
      \end{minipage}
\item Quelle(s) valeur(s) obtient-on en sortie pour chaque programme ?
\end{enumerate}
\vspace{4cm}

\textbf{Exercice 3 :}{Dans cet exercice on aborde la notion de fonction, proche de celle des maths}

\begin{enumerate}
\item \begin{enumerate}
      \item Saisir et exécuter le programme Python ci-dessous.
            
            \medskip

\begin{center}
\begin{minipage}{6cm}
\begin{lstlisting}
def equation(x):
	f = x**2-5
	g = -3*x**2+8*x+7
	return f,g
\end{lstlisting}
\end{minipage}
\end{center}

      \item Depuis la console, saisir {equation(0)}. Qu'obtient-on en sortie ?
      \item Donner une interprétation des résultats obtenus en sortie. 
      \end{enumerate}
\item À l'aide du programme, calculer les images de $f(x) = x^2-5$ et $ g(x) = -3x^2+8x+7 $ pour toutes les valeurs entières de $ x $ de $ 1 $ à $ 10 $.
\end{enumerate}

\end{document}