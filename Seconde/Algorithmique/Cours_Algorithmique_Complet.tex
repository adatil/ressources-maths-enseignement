\documentclass{article}
\usepackage[utf8]{inputenc}
\usepackage[T1]{fontenc}
\usepackage[french]{babel}
\usepackage{amsmath}
\usepackage{amssymb,amsthm}
\usepackage{geometry}
\geometry{hscale=0.85,vscale=0.85,centering}
\usepackage[dvipsnames]{xcolor}
\usepackage{listings}
\usepackage{tabularx}
\usepackage{fancyhdr}

\lstdefinestyle{pythonstyle}{
	language=Python,
	backgroundcolor=\color{gray!30},   
	commentstyle=\color{purple},
	keywordstyle=\color{blue},
	stringstyle=\color{green!70!black},
	basicstyle=\ttfamily\color{black},
	breaklines=true,
	numbers=none
}
\lstset{style=pythonstyle}

\title{Algorithmique : Débuter en Python}
\date{}

\begin{document}
\maketitle

\section{Quelques rappels}

\subsection{Qu'est-ce qu'un algorithme ?}

Un algorithme est un enchaînement d'étapes ou d'instructions à effectuer dans un certain  ordre et dont la réalisation va permettre la résolution d'un problème donné.

\medskip

\noindent Le mot algorithme vient du nom du mathématicien persan al-Khuwārizm\=ī (début du \MakeUppercase{\romannumeral9}\ieme{} siècle). Dans un livre, il décrivait des procédés de calcul à suivre étape par étape pour résoudre des problèmes ramenés à des équations.

\medskip

\noindent Un algorithme doit être lisible par tous. On l'écrit en langage courant ou langage naturel.

\medskip

\noindent Un des intérêts est de pouvoir coder (traduire) un algorithme dans un langage informatique afin qu'une machine ( ordinateur, calculatrice \ldots) puisse l'exécuter rapidement et efficacement.

\subsection{Qu'est-ce qui constitue un algorithme ?}

\begin{tabular}{p{3.7cm}p{0.2cm}p{12cm}}
\textit{Un début et une fin} &  & Les algorithmes sont constitués d'un nombre fini d'étapes à exécuter dans un ordre bien défini, on peut donc considérer qu'il y a un début et une fin. \\
\textit{Des instructions} &  & Durant l'enchaînement des étapes, ces étapes vont agir. \par On va les appeler des instructions.\\
\textit{Des variables} &  & Durant l'exécution d'un algorithme, on va avoir besoin de stocker des données, voire des résultats. Pour cela on utilise ce qu'on appelle des \og variables \fg{}. \par Le contenu d'une variable peut être modifié au cours du déroulement de l'algorithme.\\
\end{tabular}
 
\subsection{Comment analyser un algorithme ?}

Pour analyser ce que fait un algorithme on peut construire un \og tableau d'étapes \fg{}.

Un tableau d'étapes contient en première ligne (ou colonne) toutes les variables et dans  les lignes (ou colonnes) suivantes toutes les étapes d'exécution (dans l'ordre) des instructions de l'algorithme.

\section{Découvrir l'affectation}

\noindent L'affectation consiste à attribuer une valeur à une variable.

\noindent En langage courant, on écrit : Affecter à \textit{variable} la valeur \textit{valeur} ou \textit{calcul}.

\noindent Par exemple : Affecter à $ A $ la valeur $ 5 $.

\noindent On utilisera la notation suivante par la suite : $ A \gets 5 $.

\section{Programmer l'affectation}

Syntaxe des instructions utiles :

\medskip

{\renewcommand{\tabularxcolumn}[1]{%
>{{\centering\arraybackslash}}m{#1}}

\begin{center}
\begin{tabularx}{0.65\linewidth}{|X|X|}
\hline 
\textbf{Langage naturel} & \textbf{Python} \\ 
\hline 
Affecter à A la valeur 5 & A = 5 \\ 
\hline 
Saisir $ x $ & def nom\_{}fonction($ x $) \par \textit{Dans la console, on écrira :} \par nom\_{}fonction($ \ldots $) \\ 
\hline 
Afficher A & return A \par \textit{si une fonction a été définie comme ci-dessus} \\
\hline
Afficher A & print(A) \\
\hline
\end{tabularx}
\end{center}
}

\textbf{Note :} ** permet d'écrire en exposant. Par exemple, pour $ x^2 $, on écrit x**2.

\section{Exercices pratiques}

\textbf{Exercice 1 :} Programmes de base

Programmer avec Python les algorithmes suivants :

\begin{center}
\begin{minipage}{4cm}
\begin{lstlisting}
A = 2
B = 2*A
C = B**2
print(C)
\end{lstlisting}
\end{minipage}
\end{center}

\textbf{Exercice 2 :} Fonctions mathématiques

\begin{center}
\begin{minipage}{6cm}
\begin{lstlisting}
def equation(x):
    f = x**2-5
    g = -3*x**2+8*x+7
    return f,g
\end{lstlisting}
\end{minipage}
\end{center}

Ce programme permet de calculer les images de deux fonctions mathématiques.

\end{document}