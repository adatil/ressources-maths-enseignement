%\documentclass{article} %DÉSACTIVER POUR A5
\documentclass[a5paper]{article} %ACTIVER POUR A5

%########
% Packages #
%########

\usepackage[utf8]{inputenc}
\usepackage[T1]{fontenc}
\usepackage[french]{babel}

%######Affichage des maths
\DecimalMathComma %pour ne plus avoir d'espace après la virgule dans l'écriture décimale des nombres

\usepackage{amsmath}
\usepackage{amssymb,amsthm}
\usepackage{mathrsfs}
\usepackage{amsopn}

\usepackage[np]{numprint}%écriture des nombres avec des espaces et en écriture scientifique

\usepackage{dsfont} %Pour faire le 1 double barre de la fonction caractéristque dans un enironnement maths. \mathds{1}
%\usepackage{bbold} %Double barre mais en petit pour tout les nombres dans un environnement maths.\mathbb{1}

%######Graphique
\usepackage[dvipsnames]{xcolor}
\usepackage{graphicx}
\usepackage{pgf}
\usepackage{tikz}
\usepackage{tkz-tab}
\usetikzlibrary{shapes,arrows}

\usepackage{geometry} 
\geometry{hscale=0.85,vscale=0.85,centering}

%######Tableau
\usepackage{array} %pour centrer dans un tableau
\usepackage{colortbl} %pour colorier les cellules lignes colonnes d'un tableau: \rowcolor{}, \columncolor{}, \cellcolor{purple!25}
\usepackage{tabularx} %quelques amélioraions de l'environnement tabular
\usepackage{diagbox} %Pour faire une diagonale dans une case d'un tableau: \diagbox{bas gauche}{haut droit}
\usepackage{multirow} %fusionner des cellules horizontalement

%######Hyperliens dans les pdf

\usepackage[colorlinks=true,linkcolor=magenta,urlcolor=magenta]{hyperref}% Pour créer des liens à l'intérieur du pdf: \hyperlink{label}{texte du lien} permettra d'atteindre la cible identifiée par \hypertarget{label}{texte de la cible}. Les textes du lien et de la cible peuvent être vides.

%######Des symboles et images

\usepackage{marvosym} %Image de téléphone protable avec la commande \Mobilefone

\usepackage{fdsymbol} %Notamment le cœur plein: \varheartsuit

\usepackage{eurosym}%pour afficher le symbole euro

%######Vrac

\usepackage{enumerate}%énumération avec des lettres 
\usepackage{tasks}%Pour avoir une liste en ligne utiliser \begin{tasks}(2) (pour deux colonnes) et non pas enumerate puis \task et non pas \item 

\usepackage{stmaryrd}%pour faire des "intervalles" d'entiers \llbracket et \rrbracket

\usepackage{xlop}%poser les calculs en colonne: \opdiv[displayintermediary=nonzero,voperation=top,shiftdecimalsep=none]{27}{45}
\opset{decimalsepsymbol={,}}

\usepackage{verbatim}%pour utiliser commande \exclure et normalement pour faire l'affichage tel quel sans compiler le texte. 
%\usepackage{alltt}%Pour utiliser une commande latex dans un environnement verbatim il faut utiliser: alltt
%Pour écrire juste suelques mots en verbatim au milieu d'un phrase: \verb|quelques mots|

\usepackage{fancyhdr}

%######Algo

\usepackage{listings} % \begin{lstlisting} \end{lstlisting} affiche du code comme le fait le langage choisi. \lstset{language=Pascal} \lstset{language=Python} pour choisir le langage dans le document avant chaque programme ou avant le \begin{document} pour l'appliquer à tout le document. 
%\lstset{} permet d'indiquer toutes les options. Pas de caractère accentué (option lourdingue à rajouter) qui vont s'ppliquer pour toute la suite du document: \lstset{language=Python}
%Il espossible d'inclure un code python d'un fichier extérieur \lstinputlisting{source_filename.py}.
%Il est possible de définir une présentation personnalisé par un ensemble de configuration enregistré dans un fichier de style
\lstdefinestyle{pythonstyle}{
	language=Python,
	backgroundcolor=\color{gray!30},   
	commentstyle=\color{Plum},
	keywordstyle=\color{blue},
	numberstyle=\tiny\color{black},
	stringstyle=\color{ForestGreen},
	basicstyle=\ttfamily\color{black},
	breakatwhitespace=false,         
	breaklines=true,                 
	captionpos=b,                    
	keepspaces=true,                 
	numbers=none,                   
	numbersep=5pt,                  
	showspaces=false,                
	showstringspaces=false,
	showtabs=false,                  
	tabsize=1
}
\lstset{style=pythonstyle}

\lstdefinestyle{bashstyle}{
	language=bash,
	backgroundcolor=\color{black},   
	commentstyle=\color{white},
	keywordstyle=\color{magenta},
	numberstyle=\tiny\color{black},
	stringstyle=\color{white},
	basicstyle=\ttfamily\footnotesize\color{white},
	breakatwhitespace=false,         
	breaklines=true,                 
	captionpos=b,                    
	keepspaces=true,                
	numbers=left,                    
	numbersep=5pt,                  
	showspaces=false,                
	showstringspaces=false,
	showtabs=false,                  
	tabsize=1
}
%\lstset{style=bashstyle}

\usepackage[french]{algorithm2e}%pseudocode

\usepackage{scratch3}

%############### Formule developpée molécule chimie

\usepackage{chemfig}

%#####################
% Commande et environnement #
%#####################

\theoremstyle{plain}

%Pour redéfinir les commande section (changer la couleur centrer):
\usepackage{titlesec}
\titleformat{\section}[block]{\color{blue}\Large\bfseries\filcenter}{}{1em}{}
\titleformat{\subsection}[hang]{\color{purple}\large\bfseries}{\thesubsection}{1em}{}
\titleformat{\subsubsection}[hang]{\color{RoyalBlue} \bfseries}{\thesubsubsection}{1em}{}
\titleformat{\paragraph}[hang]{}{}{1em}{}

\renewcommand{\thesection}{{}}
\renewcommand{\thesubsection}{\color{purple}\Roman{subsection}}
\renewcommand{\thesubsubsection}{\color{RoyalBlue}\arabic{subsubsection}}

\newenvironment{correction}{\color{Brown}}{\medskip}

\newenvironment{sujet}{}{\medskip}

%environnement bareme
\newenvironment{bareme}{\color{RoyalBlue}\footnotesize \hfill }{\footnotesize \emph{~points}}

%environnement détais du barème
\newenvironment{details}{\color{RoyalBlue}\noindent ~\\}{~\\}

%environnement notabene
\newenvironment{notabene}{\color{PineGreen}\noindent ~\\}{~\\}

%environnement exemples
\newenvironment{exemples}{\color{blue} \medskip \noindent \underline{Exemples.}}{}

%environnement remarques
\newenvironment{remarques}{\medskip \noindent {\color{BlueViolet}\underline{Remarques.}}\color{BlueViolet}}{}

\newenvironment{lecon}{\color{black}}{}

\newenvironment{culturegenerale}{\color{Violet}}{}

%Pour redéfinir les environnements exercices et autres avec de la couleur
\newsavebox{\selvestebox}
\newenvironment{colbox}[1]
{\newcommand\colboxcolor{#1}%
	\begin{lrbox}{\selvestebox}%
		\begin{minipage}{\dimexpr\columnwidth-2\fboxsep\relax}}
		{\end{minipage}\end{lrbox}%
	\begin{center}
		\colorbox{\colboxcolor}{\usebox{\selvestebox}}
\end{center}}

%environnement exercice
\newcounter{Exercice}
\setcounter{Exercice}{1}
\newcounter{Exercicecorrection}
\newenvironment{exercice}[1]{ \setcounter{Exercicecorrection}{\theExercice} \color{black} \begin{colbox}{LimeGreen!30} \hfill \small {\color{OliveGreen}Exercice \theExercice. {\color{black}#1}} \hfill \addtocounter{Exercice}{1}}{ \end{colbox} }

%environnement exercicecorrection
\newenvironment{exercicecorrection}{\medskip \small \color{Brown} \noindent \underline{Correction de l'exercice \theExercicecorrection}
	
}{}

%environnement definition
\newcounter{Definition}
\setcounter{Definition}{1}
\newenvironment{definition}{\medskip \noindent {\color{orange}Définition \theDefinition} \addtocounter{Definition}{1} \newline \noindent \begin{tabular}{|m{12cm}|}\hline \\ }{\\  \hline \end{tabular}}

%environnement théorème il est possible d'ajouter un titre de théorème en mettant entre accolade le titre après le begin{theoreme}
\newcounter{Theoreme}
\setcounter{Theoreme}{1}
\newenvironment{theoreme}[1]{\medskip \noindent {\color{purple}Théorème \theTheoreme #1} \addtocounter{Theoreme}{1} 
	
	\noindent \begin{tabular}{|m{12cm}|}\hline \\ }{\\  \hline \end{tabular}}

%environnement proposition
\newcounter{Proposition}
\setcounter{Proposition}{1}
\newenvironment{proposition}[1]{\medskip \noindent {\color{PineGreen}Proposition \theProposition #1} \addtocounter{Proposition}{1}
	
	\noindent \begin{tabular}{|m{12cm}|}\hline \\ }{\\  \hline \end{tabular}}

%environnement propriété
\newcounter{Propriete}
\setcounter{Propriete}{1}
\newenvironment{propriete}[1]{\medskip \noindent {\color{PineGreen}Propriété \thePropriete #1} \addtocounter{Propriete}{1}
	
	\noindent \begin{tabular}{|m{12cm}|}\hline \\ }{\\  \hline \end{tabular}}

%environnement méthode
\newcounter{Methode}
\setcounter{Methode}{1}
\newenvironment{methode}[1]{\medskip \noindent {\color{PineGreen}Méthode \theMethode #1} \addtocounter{Methode}{1}
	
	\noindent \begin{tabular}{|m{12cm}|}\hline \\ }{\\  \hline \end{tabular}}

%environnement lemme
\newcounter{Lemme}
\setcounter{Lemme}{1}
\newenvironment{lemme}[1]{\medskip \noindent {\color{PineGreen}Lemme \theLemme #1} \addtocounter{Lemme}{1}
	
	\noindent \begin{tabular}{|m{12cm}|}\hline \\ }{\\  \hline \end{tabular}}

%environnement corollaire
\newcounter{Corollaire}
\setcounter{Corollaire}{1}
\newenvironment{corollaire}[1]{\medskip \noindent {\color{PineGreen}Corollaire \theCorollaire #1} \addtocounter{Corollaire}{1} 
	
	\noindent \begin{tabular}{|m{12cm}|}\hline \\ }{\\  \hline \end{tabular}}

%environnement démonstration
\newcounter{Demonstration}
\setcounter{Demonstration}{1}
\newenvironment{preuve}[1]{\medskip \noindent {\color{PineGreen} Démonstration} \hfill #1 \addtocounter{Demonstration}{1} \color{violet} 
	
}{\hfill $\blacksquare$}

%environnement conclusion encadré et coloré
\newenvironment{conclusion}
{\color{PineGreen}\begin{tabular}{|c|}\hline \\ \begin{minipage}{0.85\linewidth} \begin{center} }
			{\end{center} \end{minipage} \\ \\ \hline \end{tabular} }

%Commande pour l'objectif et l'écrire en vert
\newcommand{\objectif}[1]{{\color{PineGreen}#1}
	
	\medskip}

\usepackage{fancyhdr}

\pagestyle{empty}
\pagestyle{fancy} 
\renewcommand{\headrulewidth}{0pt}%Pas de ligne horizontale en haut
\lhead[]{}%entre crochets pages paires entre accolades pages impaires
\chead[\small ]{\footnotesize \href{http://unemainlavelautre.net/2ieme.html}{03 Introduction à Python.} }% l left, c center, r right
\rhead[]{}
\lfoot[]{}
\cfoot[\small -\thepage -]{\small -\thepage -}
\rfoot[]{}

\begin{document}

\section{03 Introduction à Python.}

\subsection{Variables.}

\begin{lecon}
	
	Les objets élémentaires que l'on peut manipuler en Python sont les entiers, les \emph{\color{purple}booléens} (une phrase dont on peut dire si elle est vraie ou fausse), les \emph{\color{purple}flottants} (des nombres décimaux ou en écriture scientifiques pour les très grands nombres) et  les \emph{\color{purple}chaînes de caractères} (des mots).
	
	On appelle \emph{\color{purple}variable} une lettre ou un mot qui désigne un entier, un booléen, un flottant ou une chaîne de caractère.
	
\end{lecon}

\subsection{Opérations en Python.}

\begin{lecon}
	
	Puisqu'il y a des nombres on retrouve les opérations. Addition, soustraction, multiplication, et division se notent $+$, $-$, $*$ et $/$.
	
	La puissance se note $**$.
	
	Il existe d'autres opérations que nous rencontrerons moins souvent: $//$ (quotient de la division euclidienne), $\%$ reste de la division euclidienne.
	
\end{lecon}

\subsection{Affectation, séquences.}

\begin{lecon}
	
	L'affectation est le fait d'enregistrer dans une variable un entier, un booléen, un flottant ou une chaîne de caractère. En Python le symbole pour l'affectation est $=$. Ainsi $a=3$ signifie que la variable $a$ prend la valeur $3$. Dans certains exercices l'affectation, mais pas en Python, est notée $a \leftarrow 3$.
	
\end{lecon}

\begin{lecon}
	
	\medskip
	Une séquence est une succession de plusieurs commandes. Pour créer une nouvelle commande il suffit de passer à la ligne.
	
\end{lecon}

\begin{lecon}
	
	\medskip
	Pour choisir entre deux commandes possibles on utilise une instruction conditionnelle. Si une condition est vraie (un booléen) alors il faut effectuer une certaine commande. 
	
\end{lecon}

\subsection{Exercices.}

\begin{exercice}{A}
	
	Déterminez la valeur renvoyée par le programme de calcul en prenant pour valeur initiale la valeur $a$ proposée. Puis proposez une formule algébrique en fonction de $a$ exprimant le résultat renvoyé par le programme.
	
	\begin{tasks}(1)
		\task $a=1$ et 
		\begin{tabular}{|m{6cm}|}
			\hline
			Choisir un nombre.
			\\
			Ajouter $2$ à ce nombre.
			\\
			Multiplier le précédent résultat par $4$.
			\\
			Mettre le résultat au carré.
			\\
			\hline
		\end{tabular}
		\task $a=-3$ et 
		\begin{tabular}{|m{6cm}|}
			\hline
			Choisir un nombre.
			\\
			Ajouter $3$.
			\\
			Multiplier le résultat par le nombre choisi.
			\\
			Soustraire $16$.
			\\
			\hline
		\end{tabular}
	\end{tasks}
	
\end{exercice}

\end{document}