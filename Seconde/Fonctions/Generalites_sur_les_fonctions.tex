\documentclass{article}
\usepackage[utf8]{inputenc}
\usepackage[T1]{fontenc}
\usepackage[french]{babel}
\usepackage{amsmath}
\usepackage{amssymb,amsthm}
\usepackage{mathrsfs}
\usepackage{amsopn}
\usepackage[np]{numprint}
\usepackage{dsfont}
\usepackage[dvipsnames]{xcolor}
\usepackage{graphicx}
\usepackage{pgf}
\usepackage{tikz}
\usepackage{tkz-tab}
\usetikzlibrary{shapes,arrows}
\usepackage{pgfplots}
\usepackage{geometry}
\geometry{hscale=0.85,vscale=0.85,centering}
\usepackage{array}
\usepackage{colortbl}
\usepackage{tabularx}
\usepackage{diagbox}
\usepackage{multirow}
\usepackage[colorlinks=true,linkcolor=magenta,urlcolor=magenta]{hyperref}
\usepackage{marvosym}
\usepackage{fdsymbol}
\usepackage{eurosym}
\usepackage{enumerate}
\usepackage{tasks}
\usepackage{stmaryrd}
\usepackage{xlop}
\usepackage{verbatim}
\usepackage{fancyhdr}
\usepackage{listings}
\usepackage[french]{algorithm2e}
\usepackage{scratch3}
\usepackage{chemfig}
\usepackage{titlesec}

% Commandes math\u00e9matiques
\newcommand{\vect}[1]{\mathchoice%
	{\overrightarrow{\displaystyle\mathstrut#1\,\,}}%
	{\overrightarrow{\textstyle\mathstrut#1\,\,}}%
	{\overrightarrow{\scriptstyle\mathstrut#1\,\,}}%
	{\overrightarrow{\scriptscriptstyle\mathstrut#1\,\,}}}
\newcommand{\R}{\mathbb{R}}
\newcommand{\N}{\mathbb{N}}
\newcommand{\D}{\mathbb{D}}
\newcommand{\Z}{\mathbb{Z}}
\newcommand{\Q}{\mathbb{Q}}
\newcommand{\C}{\mathbb{C}}

% Formatage des sections
\titleformat{\section}[hang]{\color{blue}\Large\bfseries}{\thesection}{1em}{}
\titleformat{\subsection}[hang]{\color{purple}\large\bfseries}{\thesubsection}{1em}{}
\titleformat{\subsubsection}[hang]{\color{RoyalBlue} \bfseries}{\thesubsubsection}{1em}{}

\renewcommand{\thesection}{\color{blue}\Alph{section}}
\renewcommand{\thesubsection}{\color{purple}\Roman{subsection}}
\renewcommand{\thesubsubsection}{\color{RoyalBlue}\arabic{subsubsection}}

% Environnements color\u00e9s
\newsavebox{\selvestebox}
\newenvironment{colbox}[1]
{\newcommand\colboxcolor{#1}%
	\begin{lrbox}{\selvestebox}%
		\begin{minipage}{\dimexpr\columnwidth-2\fboxsep\relax}}
		{\end{minipage}\end{lrbox}%
	\begin{center}
		\colorbox{\colboxcolor}{\usebox{\selvestebox}}
\end{center}}

% Environnement d\u00e9finition
\newcounter{Definition}
\setcounter{Definition}{1}
\newenvironment{definition}{\medskip \noindent {\color{orange}D\u00e9finition \theDefinition} \addtocounter{Definition}{1} \newline \noindent \begin{tabular}{|m{14cm}|}\hline \\ }{\\  \hline \end{tabular}}

% Environnement remarques
\newenvironment{remarques}{\medskip \noindent {\color{BlueViolet}\underline{Remarques.}}\color{BlueViolet}}{}

% Environnement exemples
\newenvironment{exemples}{\color{blue} \medskip \noindent \underline{Exemple.}}{}

% Environnement m\u00e9thode
\newcounter{Methode}
\setcounter{Methode}{1}
\newenvironment{methode}[1]{\medskip \noindent {\color{PineGreen}M\u00e9thode \theMethode #1} \addtocounter{Methode}{1}
	
	\noindent \begin{tabular}{|m{16cm}|}\hline \\ }{\\  \hline \end{tabular}}

\title{G\u00e9n\u00e9ralit\u00e9s sur les Fonctions}
\date{}

\begin{document}
\maketitle

\paragraph*{Objectif du chapitre :}
\begin{enumerate}
\item[\u2022] Reconna\u00eetre une fonction d\u00e9finie par un processus, une courbe, un tableau de valeurs.
\item[\u2022] D\u00e9terminer l'image (ou un ant\u00e9c\u00e9dent) d'un nombre par une fonction connue gr\u00e2ce \u00e0 son expressions litt\u00e9rale, sa courbe repr\u00e9sentative ou son tableau de valeurs
\end{enumerate}

\section{Les fonctions en classe de troisi\u00e8me}

\begin{definition}
D\u00e9finir une \textbf{fonction} sur une partie $D$ de l'ensemble des nombres r\u00e9els $\R$, c'est associer \u00e0 tout nombre $x$ de $D$ un unique nombre r\u00e9el $y$.

\noindent On note \quad $f \colon x \mapsto y$ \quad ou \quad $y = f(x)$

\noindent On dit que $x$ est la \textbf{variable}.
\end{definition}

\begin{remarques}
Une fonction $f$ d\u00e9finie sur $D$ peut \u00eatre donn\u00e9e de trois fa\u00e7ons: par une \textbf{formule} ou une \textbf{expression alg\u00e9brique}, par une \textbf{courbe repr\u00e9sentative} ou encore par un \textbf{tableau de valeurs}.           
\end{remarques}

\section{Vocabulaire}

\begin{definition}
Une \textbf{fonction} est un proc\u00e9d\u00e9 qui \u00e0 un nombre $x$ appartenant \u00e0 un ensemble $\mathcal{D}$ associe un nombre $y$.
   
\noindent On dit que $y$ est l'\textbf{image} de $x$ par la fonction $f$ 
   
\noindent On dit que $x$ est un \textbf{ant\u00e9c\u00e9dent} de $y$ par la fonction $f$.
\end{definition}

\begin{remarques}
Pour toute fonction $f$, un nombre $x$ a une et une seule image par $f$. 

\noindent Par contre, chaque nombre $y$ peut avoir plusieurs ant\u00e9c\u00e9dents, ou ne pas avoir d'ant\u00e9c\u00e9dents.
\end{remarques}

\begin{exemples}
   Soit $g$ la fonction d\u00e9finie par $g(x)=x^2+3$.
   \begin{enumerate}
      \item[\u2022] L'image de $5$ est $g(5)=5^2+3=28$,
      \item[\u2022] Les ant\u00e9c\u00e9dents de $7$ v\u00e9rifient $g(x)=7$ c'est \u00e0 dire $x^2+3=7$ soit $x=-2$ ou $x=2$,
      \item[\u2022] Il n'y a pas d'ant\u00e9c\u00e9dent de $1$ car l'\u00e9quation $g(x)=1$ n'a pas de solution : $x^2+3=1 \Longleftrightarrow x^2=-2$.
   \end{enumerate}
\end{exemples}

\begin{definition}
   Pour une fonction $f$ donn\u00e9e, l'ensemble de tous les nombres r\u00e9els qui ont une image calculable par cette fonction est appel\u00e9 \textbf{ensemble de d\u00e9finition} de la fonction $f$, que l'on notera $\mathcal{D}_f$.
\end{definition}

\medskip
Graphiquement, l'ensemble de d\u00e9finition est l'intervalle sur lequel la courbe existe.

\begin{exemples}
La fonction $f:x \mapsto \dfrac{1}{2x-4}$ a pour ensemble de d\u00e9finition $] - \infty ; 2\; [\; \cup ]\;2 ;+ \infty [$.
\begin{enumerate}
\item[\u2022] En effet, l'expression $\dfrac{1}{2x-4}$ n'a de sens que pour les valeurs de $x$ telles que $2x-4\neq 0$ (car le d\u00e9nominateur d'une fraction ne peut \u00eatre \u00e9gal \u00e0 0), c'est-\u00e0-dire pour $x \neq 2$,
\item[\u2022] On dira aussi que $2$ est une \textbf{valeur interdite} pour la fonction $f$.
\end{enumerate}
\end{exemples}

\section{Tableau de valeurs}

Pour une fonction $f$ donn\u00e9e, on peut \u00e9tablir un tableau de valeurs.
Dans ce tableau, la premi\u00e8re ligne contient des nombres r\u00e9els $x$, et la seconde ligne contient leurs images respectives $y$.

\begin{exemples}
Soit la fonction $f$ d\u00e9finie sur $\R^*$ par $f(x)=x+\dfrac{2}{x}$, \\
on obtient le tableau suivant (gr\u00e2ce par exemple \u00e0 une calculatrice) :

\renewcommand{\arraystretch}{1.8}
\begin{tabular}{|c||c||c||c||c||c||c||c||c||c||c|} 
\hline
$x$ & $-4$ & $-3$ & $-2$ & $-1$ & $0$ & $1$ & $2$ & $3$ \\\hline
$f(x)$ & $-4,7$ & $-3,7$ & $-3$ & $-3$ & & $3$ & $3$ & $3,7$ \\\hline
\end{tabular}

\noindent On remarque que dans la ligne des \u00ab y \u00bb, certaines cases peuvent rester vides. En effet, certaines fonctions n'ont pas d'image pour des valeurs de \u00ab x \u00bb 
\end{exemples}

\section{Courbe repr\u00e9sentative}

\begin{definition}
Dans un rep\u00e8re $(O,I,J)$, l'ensemble des points $M$ de coordonn\u00e9es $\left(x;f(x)\right)$ forme la \textbf{courbe repr\u00e9sentative de la fonction $f$}, souvent not\u00e9e $\mathcal{C}_f$.
\end{definition}

\begin{methode}{Construction d'une courbe repr\u00e9sentative}
On souhaite tracer la courbe repr\u00e9sentative de la fonction $f$ d\u00e9finie sur $[-3;2]$ par : $f(x)=\dfrac{5x}{x^2+1}$.

\begin{enumerate}
\item On commence par compl\u00e9ter un tableau de valeurs :
\begin{center}
\renewcommand{\arraystretch}{2}
\begin{tabular}{|c||c||c||c||c||c||c||c||c||c||c||c|}
\hline
$x$ & $-3$ & $-2,5$ & $-2$ & $-1,5$ & $-1$ & $-0,5$ & $0$ & $0,5$ & $1$ & $1,5$ & $2$ \\\hline
$f(x)$& $-1,5$ & $-1,7$ & $-2$ & $-2,3$ & $-2,5$ & $-2$ & $0$ & $2$ & $2,5$ & $2,3$ & $2$ \\\hline
\end{tabular}
\end{center}

\item Puis on place les points $M\left(x;f(x)\right)$ dans un rep\u00e8re
\item On relie les points pour obtenir la courbe
\end{enumerate}
\end{methode}

\begin{remarques}
Le point de coordonn\u00e9es $(10;0.5)$ n'est pas sur la courbe repr\u00e9sentative de la fonction $f$ car $f(10)=0,495 \neq 0,5$.
\end{remarques}

\section{Lecture graphique}

\textbf{M\u00e9thode pour lire une image :}
\begin{enumerate}
\item[\u2022] On place $x$ sur l'axe des abscisses
\item[\u2022] On se d\u00e9place verticalement pour rencontrer $\mathcal{C}_f$
\item[\u2022] On lit $f(x)$ sur l'axe des ordonn\u00e9es
\end{enumerate}

\textbf{M\u00e9thode pour trouver un ant\u00e9c\u00e9dent :}
\begin{enumerate}
\item[\u2022] On trace une horizontale passant par la valeur recherch\u00e9e
\item[\u2022] On note les points d'intersection avec $\mathcal{C}_f$
\item[\u2022] On se d\u00e9place verticalement vers l'axe des abscisses pour lire les ant\u00e9c\u00e9dents
\end{enumerate}

\end{document}